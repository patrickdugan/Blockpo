\documentclass[preprint,reqno,12pt]{elsarticle}

\usepackage[latin1]{inputenc} 
\usepackage{amsmath} 
\usepackage{amsfonts} 
\usepackage{amssymb} 
\usepackage{amsthm} 
\usepackage{graphicx}  
\usepackage{latexsym}    

\usepackage[T1]{fontenc} 

\usepackage{amsmath}

\usepackage{graphicx}
\usepackage{amsfonts}
\def\fl{\Rightarrow}
\def\px{\partial_x}
\def\pt{\partial_t}
\newtheorem{Alg}{Algorithm}[section]
\newtheorem{example}{Example}[section]

\newtheorem{corollary}{Corollary}[section]
\newtheorem{theorem}{Theorem}[section]
\newtheorem{lemma}{Lemma}[section]
\theoremstyle{definition}
\newtheorem{remark}{Remark}[section]
\newtheorem{definition}{Definition}[section]
\newtheorem*{definition*}{Definition}
\newproof{pf}{Proof}

\newcommand{\Feps}{\boldsymbol{F}_{\varepsilon}} 
\newcommand{\sgn}{\operatorname*{sgn}}
\newcommand{\N}{{\ensuremath{\mathbb{N}}}}
\newcommand{\Z}{{\ensuremath{\mathbb{Z}}}}
\newcommand{\Q}{{\ensuremath{\mathbb{Q}}}}
\newcommand{\R}{{\ensuremath{\mathbb{R}}}}
\newcommand{\C}{{\ensuremath{\mathbb{C}}}}
\newcommand{\prom}{{\displaystyle \int \!\!\!\!\!\! - }}
\newcommand{\pp}[2] {\fracd{\partial {#1}}{\partial {#2}}}
\newcommand{\tr}{\operatorname*{tr}}

\newcommand{\bbc}{{\boldsymbol c}}
\newcommand{\bbw}{{\boldsymbol w}}
\def\rpn{\mathbb{R}_+^N}
\newcommand{\bw}{\boldsymbol w}
\newcommand{\bLf}{\boldsymbol{\mathfrak{L}}}
\newcommand{\bW}{\boldsymbol W}
\newcommand{\bL}{\boldsymbol L}
\newcommand{\bc}{\boldsymbol c}
\newcommand{\bC}{\boldsymbol C}
\newcommand{\bx}{\boldsymbol x}
\newcommand{\bb}{\boldsymbol b}
\newcommand{\bq}{\boldsymbol q}
\newcommand{\bl}{\boldsymbol l}
\newcommand{\bs}{\boldsymbol s}
\newcommand{\minmod}{\operatorname*{minmod}} 
\newcommand{\diag}{\operatorname*{diag}}
\newcommand{\bt}{\boldsymbol \tau}
\newcommand{\clC}{{\boc{C}}}

\def\eps{\varepsilon}  
   
\usepackage{graphicx}
\usepackage[british,UKenglish,USenglish,english,american]{babel}
   
\usepackage[usenames]{color}


\newcommand{\bfj}{\boldsymbol{\hat f}}
\def\mig{\frac12}  
\def\refe{\text{ref}}
\def\boc#1{\boldsymbol{{\cal #1}}}
\def\mD{\boc{D}}
\def\bh#1{\boldsymbol{\hat{#1}}}
\usepackage{xcolor}
\usepackage[normalem]{ulem}
\numberwithin{equation}{section}


\begin{document}

\begin{frontmatter}  
   
\title{Clearing Algorithm and Shortest Path for Settlement in Futures Contracts}    

\author[pd]{Patrick Dugan} \ead{likyalpha@gmail.com} 
\author[dp]{Daniel Pizarro} \ead{likyalpha@gmail.com} 
\author[rb]{Lihki Rubio} \ead{likyalpha@gmail.com} 
  
\cortext[cor]{Corresponding author.}   
  
\address[pd]{Tradelayer, Conc\'{o}n, Valparaiso, Chile}   
  
\selectlanguage{USenglish}
\begin{abstract}

Centralization in cryptocurrency derivatives has been an issue for a some years, since exchanges have to be compliant to have their bank accounts in order, and they have to provide the required KYC and AML. The main issue is they work as non-deliverable, multilateral protocols, of which security depends on a cold and hot storage of tokens with centralized security protocols. To create a decentralized solution to this, where there is no central point of failure, it needs for a compliant protocol where settlement is bilateral. This creates a problem of using settlement algorithms, because other than working fast, it has to have hard-coded delivery of tokens, and an all time overview of open interest. We propose a strategy based on graphs, where each participant is represented by a vertice, connected by amounts as their respective edges. Following a flow mechanism, and a continuous accounting, we can ameliorate counterparty risk, and deliver in-protocol derivatives as future contracts at settlement.

\end{abstract} 
   
\begin{keyword} 
Financial derivatives, Clearing Algorithm, Shortest Path  
\end{keyword}
 
\end{frontmatter} 

\pagenumbering{arabic}

\selectlanguage{USenglish}
Here start the section for clearing algo. Citation example: \cite{kenyon2002high}.

\begin{definition*}[\textbf{Futures Positions: Long and Short}]
  A futures contract is a commitment between two or more counterparties who agree to
  engage in a transaction at a later date and a prespecified price. The contract gives
  one party the right to buy/sell the underlying asset for a specific price at a specific
  date in the future. We call this specific price a futures price.
  Depending on whether we acquire the right to buy or sell, we refer to our
  position as a long or a short position, respectively. The value of the contract at
  the outset is zero as both counterparties have the same probability of making
  gains at that point. No cash changes hands as a result.
  Thus, the long position represents the right to buy the underlying asset at a
  specific futures price. The short position gives its holder the right to sell the
  underlying asset at a futures price sometime in the future. Futures (and
  forward) contracts can be written on a variety of underlying assets such as
  commodities, foreign exchange, short-term debt, and stock indices.
\end{definition*}

\begin{definition}[\textbf{Status Futures Positions}]

Let $\bLf$ be a set of characters and\, $\bl=(l_{1}, l_{2}, l_{3}, l_{4}),\bs=(s_{1}, s_{2}, s_{3}, s_{4})\in\bLf^4$
status for future postions defined by mean

\begin{align}
\begin{bmatrix}
        \,l_{1} \\
        \,l_{2} \\
        \,l_{3} \\
        \,l_{4}
\end{bmatrix}
&=
\begin{bmatrix}
        \textsc{Long Position Increased} \\
        \textsc{Open Long Position} \\
        \textsc{Long Position Netted Partly\,\,} \\
        \textsc{Long Position Netted} \\ 
\end{bmatrix}
\\
\begin{bmatrix}
        s_{1} \\
        s_{2} \\
        s_{3} \\
        s_{4}
\end{bmatrix}
&=
\begin{bmatrix}
        \textsc{Short Position Increased} \\
        \textsc{Open Short Position} \\
        \textsc{Short Position Netted Partly} \\
        \textsc{Short Position Netted} \\ 
\end{bmatrix}
\end{align}

\end{definition}


\selectlanguage{USenglish}
Here start the section for applications. Citation example: \cite{sacks2015elementary}, \cite{science2010network}.


\selectlanguage{USenglish}
Here start the conclusions. Citation example: \cite{kwok2008mathematical}.


\newpage

\selectlanguage{USenglish}
\bibliographystyle{ieeetr}
\bibliography{paper1bib} 
  
\end{document}
\grid
